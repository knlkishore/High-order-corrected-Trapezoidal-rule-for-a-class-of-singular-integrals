\documentclass{article}
\usepackage[utf8]{inputenc}
\usepackage{amsthm}
\usepackage{amsmath}
\usepackage{mathtools}
\title{Lagrange Interpolation for Equally-Spaced abscissa}
\author{Kunal Kishore }
\date{}


%-----------------------------------------------
\parindent 0pt

\newtheorem*{theorem*}{Theorem}
\newtheorem{theorem}{Theorem}

%------------------------------------------------
\begin{document}

\maketitle


\begin{theorem*}[Lagrange Interpolation for Equally-Spaced abscissa]

Suppose $a,b$ are a pair of real number such that $a<b$ , $m \ge 3$ be an integer and $h = \frac{b-a}{m-1}$. 

let $f \in C^{m}[a-mh,b+mh]$ and equispaced points $x_k$ be defined as  $x_k  = \frac{b-a}{2} +kh$.
Then, for any real number $p$ there exists a real number 
$\xi , -mh <\xi<mh $, such that

\begin{equation}
    f(x_0 + ph) = \sum_{k} \mathcal{A}_{k}^{m} f(x_k) + \mathcal{R}_{m-1,p}
\end{equation}
where $k$ varies from

% \begin{align*}
%     \frac{1}{2} (m-2) \leq k \leq \frac{1}{2}m &  \quad 
%     \textit{for even m}\\
%     \frac{1}{2} (m-1) \leq k \leq \frac{1}{2}(m-1) & \quad 
%     \textit{for odd m}
% \end{align*}


$$
-\frac{1}{2}(m-2) \leq k \leq \frac{1}{2}m  \quad 
\textit{for even m}
$$
$$
-\frac{1}{2}(m-1) \leq k \leq \frac{1}{2}(m-1)  \quad 
\textit{for odd m}
$$

with
$$
\mathcal{A}_{k}^{m}(p) = 
\frac{ (-1)^{\frac{m-1}{2}+k} }{
(\frac{m-1}{2} + k)!
(\frac{m-1}{2} - k)!
(p-k)}
\sum_{t}(p-t)
$$
and
$$
\mathcal{R}_{m-1,p} = \frac{ h^{m} }{m!} f^{(m)}(\xi)
\sum_{n} (p-n)
$$
here both $t$ and $n$ varies same as $k$. 
\end{theorem*}
%---------------------------------------------------------
\begin{proof}
Assuming $m$ to be odd.
For a function $f \in C^{m}[a-mh,b+mh]$ ,its Lagrange interpolation polynomial constructed from  equispaced points 
$$
x_{-\frac{1}{2}(m-1)} , x_{-\frac{1}{2}(m-1)+1} , .......
, x_{\frac{1}{2}(m-1)-1} , x_{\frac{1}{2}(m-1)}
$$


% \begin{equation*} \label{nodes}
% \begin{split}
% \Bigg[
% \left(-\frac{1}{2}(m-1) , f(-\frac{1}{2}(m-1)) \right) , 
% \left(-\frac{1}{2}(m-1) +1 , f(-\frac{1}{2}(m-1) +1 ) \right),\\
% ......
% ,\left(\frac{1}{2}(m-1), f(\frac{1}{2}(m-1)) \right)
% \Bigg]
% \end{split}
% \end{equation*}


is given by


\begin{equation} 
    \mathcal{L}_{m-1}(x) = 
    \sum_{ j=-\frac{1}{2}(m-1) }^{ \frac{1}{2}(m-1) } 
    l_j(x_j) f(x_j)
\end{equation}
and,

\begin{equation} \label{main_eqn}
f(x) =  
\sum_{ j=-\frac{1}{2}(m-1) }^{ \frac{1}{2}(m-1) } 
l_j(x_j) f(x_j) + R_{m-1}(x)
\end{equation}


with,
\begin{equation*} 
    l_j(x) = 
    \sum_{i = -\frac{1}{2}(m-1) , i \neq j}^{\frac{1}{2}(m-1)} 
    \frac{x - x_i}{x - x_j}
\end{equation*}
    
\begin{equation*} 
    R_{m-1}(x) =  \frac{ f^{(m)}(\xi) }{ m! }
    \bigg[  
    \sum_{j = -\frac{1}{2}(m-1) }^{ \frac{1}{2}(m-1) } 
    x-x_j
    \bigg]
\end{equation*}



We can write $x$ as $x = x_0 + ph $ for some real number $p$.
% Assuming $m$ to be odd.
% Constructing the Lagrange interpolation polynomial at equispaced points 
% \begin{equation*} \label{nodes}
% \begin{split}
% \Bigg[
% \left(-\frac{1}{2}(m-1) , f(-\frac{1}{2}(m-1)) \right) , 
% \left(-\frac{1}{2}(m-1) +1 , f(-\frac{1}{2}(m-1) +1 ) \right),\\
% ......
% ,\left(\frac{1}{2}(m-1), f(\frac{1}{2}(m-1)) \right)
% \Bigg]
% \end{split}
% \end{equation*}

\begin{equation}   \label{remainder_final}
\begin{split}
R_{m-1}(x) &=  R_{m-1}(x_0 + ph)  
    =\frac{ f^{(m)}(\xi) }{ m! }
        \bigg[  
        \sum_{j = -\frac{1}{2}(m-1) }^{ \frac{1}{2}(m-1) } 
        x-x_j
        \bigg] 
    \\
    &=\frac{ f^{(m)}(\xi) }{ m! }
        \bigg[  
        \Big(x - x_{-\frac{1}{2}(m-1)} \Big)
        \Big(x - x_{-\frac{1}{2}(m-1)+1} \Big)
        .......
        \Big(x - x_{\frac{1}{2}(m-1)} \Big)
        \bigg]  
    \\
    &=\frac{ f^{(m)}(\xi) }{ m! }
        \bigg[  
            \Big(
            x_0 +ph -x_0 + \frac{1}{2}(m-1)h
            \Big)
            ....
            \Big(
            x_0 +ph -x_0 - \frac{1}{2}(m-1)h
            \Big)
        \bigg] 
    \\
    &=\frac{ f^{(m)}(\xi) }{ m! }
        \bigg[  
            \Big(
            ph + \frac{1}{2}(m-1)h
            \Big)
            ....
            \Big(
            ph- \frac{1}{2}(m-1)h
            \Big)
        \bigg] 
    \\
    &=\frac{ f^{(m)}(\xi) }{ m! }
        \bigg[  
            h^{m}
            \Big(
            p + \frac{1}{2}(m-1)
            \Big)
            ....
            \Big(
            p- \frac{1}{2}(m-1)
            \Big)
        \bigg] 
    \\
    &=\frac{ f^{(m)}(\xi) }{ m! }
        \bigg[  
            h^{m}
            \Big(
            p + \frac{1}{2}(m-1)
            \Big)
            ....
            \Big(
            p- \frac{1}{2}(m-1)
            \Big)
        \bigg] 
    \\
    &=\frac{ f^{(m)}(\xi) }{ m! } h^{m}
      \sum_{n = -\frac{1}{2}(m-1)}^{\frac{1}{2}(m-1)}
      (p-n)
    \\
    &=\mathcal{R}_{m-1,p}
\end{split}
\end{equation}



\begin{equation*}
\begin{split}
    l_k(x) &=  l_k( x_0 + ph ) =
    \sum_{i = -\frac{1}{2}(m-1) , i \neq k}^{\frac{1}{2}(m-1)} 
    \frac{x - x_i}{x_k - x_i} 
    \\
    &=\Bigg[\frac{
        \big(x  - x_{-\frac{1}{2}(m-1)} \big)
        ......
        \big(x  - x_{k-1} \big)
        \big(x  - x_{k+1} \big)
        .......
        \big(x - x_{\frac{1}{2}(m-1)} \big)
    }{
        \big(x_k - x_{-\frac{1}{2}(m-1)} \big)
        ......
        \big(x_k  - x_{k-1} \big)
        \big(x_k  - x_{k+1} \big)
        .......
        \big(x - x_{\frac{1}{2}(m-1)} \big)
    }
    \Bigg]
\end{split}
\end{equation*}

\begin{equation}        \label{alpha_beta}
\begin{split}
    &=\left[ \frac{
                \splitfrac{
                \Big(
                x_0 +ph -x_0 + \frac{1}{2}(m-1)h
                \Big)
                ....
                \Big(
                x_0 +ph -x_0 - (k-1)h
                \Big)
                 \Big(
                x_0 +ph -x_0 - (k+1)h
                \Big)
                }{
                ....
                \Big(
                x_0 +ph -x_0 - \frac{1}{2}(m-1)h
                \Big)
                }
    }{
                \splitfrac{
                \Big(
                x_0 +kh -x_0 + \frac{1}{2}(m-1)h
                \Big)
                ....
                \Big(
                x_0 +kh -x_0 - (k-1)h
                \Big)
                 \Big(
                x_0 +kh -x_0 - (k+1)h
                \Big)
                }{
                ....
                \Big(
                x_0 +kh -x_0 - \frac{1}{2}(m-1)h
                \Big)
                }
    }
    \right]
    \\
    % &=\left[ \frac{
    %             \Big(
    %             ph + \frac{1}{2}(m-1)h
    %             \Big)
    %             ....
    %             \Big(
    %             ph - \frac{1}{2}(m-1)h
    %             \Big)
    %             \Big(
    %             ph  - \frac{1}{2}(m-1)h
    %             \Big)
    %             ....
    %             \Big(
    %             ph  - \frac{1}{2}(m-1)h
    %             \Big)
    % }{
    %             \Big(
    %             ph + \frac{1}{2}(m-1)h
    %             \Big)
    %             ....
    %             \Big(
    %             ph - \frac{1}{2}(m-1)h
    %             \Big)
    %              \Big(
    %             ph  - \frac{1}{2}(m-1)h
    %             \Big)
    %             ....
    %             \Big(
    %             ph  - \frac{1}{2}(m-1)h
    %             \Big)
    % }
    % \right]
    % \\
    &=\left[ \frac{
                \Big(
                p + \frac{1}{2}(m-1)
                \Big)
                ....
                \Big(
                p - (k-1)
                \Big)
                \Big(
                p  - (k+1)
                \Big)
                ....
                \Big(
                p  - \frac{1}{2}(m-1)
                \Big)
    }{
                \Big(
                k + \frac{1}{2}(m-1)
                \Big)
                ....
                \Big(
                k - (k-1)
                \Big)
                 \Big(
                k  - (k+1)
                \Big)
                ....
                \Big(
                k  - \frac{1}{2}(m-1)
                \Big)
    }
    \right] \frac{\Big(p-k\Big)}{\Big(p-k\Big)}
\\
   &=\left[ \frac{
                \Big(
                p + \frac{1}{2}(m-1)
                \Big)
                ....
                \Big(
                p - k+1
                \Big)
                \Big(p-k\Big)
                \Big(
                p  - k-1
                \Big)
                ....
                \Big(
                p  - \frac{1}{2}(m-1)
                \Big)
    }{
                \Big(
                k + \frac{1}{2}(m-1)
                \Big)
                ....
                \Big(
                k - k + 1
                \Big)
                 \Big(
                k  - k -1
                \Big)
                ....
                \Big(
                k  - \frac{1}{2}(m-1)
                \Big)
    }
    \frac{1}{  \Big(p-k\Big)}
    \right] 
\\
    &=\frac{ \alpha }{ \beta }
\end{split}
\end{equation}

where 
$\alpha = 
            \big(
                p + \frac{1}{2}(m-1)
                \big)
                ....
                \big(
                p - k+1
                \big)
                \big(p-k\big)
                \big(
                p  - k-1
                \big)
                ....
                \big(
                p  - \frac{1}{2}(m-1)
            \big)
            % = \prod_{t = -\frac{1}{2}(m-1) }^{\frac{1}{2}(m-1)}  (p-t)
$
, $\beta =   \big(
                k + \frac{1}{2}(m-1)
                \big)
                ....
                \big(
                k - k + 1
                \big)
                \big(
                k  - k -1
                \big)
                ....
                \big(
                k  - \frac{1}{2}(m-1)
                \big)
                \big(p-k\big)$
    
Now,
\begin{equation*}
\begin{split}
    \beta   &= \big(
                k + \frac{1}{2}(m-1)
                \big)
                ....
                \big(
                k - k + 1
                \big)
                \big(
                k  - k -1
                \big)
                ....
                \big(
                k  - \frac{1}{2}(m-1)
                \big)
                \big(p-k\big)
    \\
         &= \big(
                k + \frac{1}{2}(m-1)
                \big)
                ....
                \big(  2  \big)
                \big(  1  \big)
                \big( -1  \big)
                \big( -2  \big)
                ....
                \big(
                k  - \frac{1}{2}(m-1)
                \big)
                \big(p-k\big)
    \\
         &= \Big( k + \frac{1}{2}(m-1) \Big)! 
                \big( -1  \big)
                \big( -2  \big)
                ....
                \big(
                k  - \frac{1}{2}(m-1)
                \big)
                \big(p-k\big)
    \\
         &= \Big( k + \frac{1}{2}(m-1) \Big)! 
            \Big(  -1  \Big)^{\frac{m-1}{2}-k}    
                \big( 1  \big)
                \big( 2  \big)
                ....
                \big(
                \frac{1}{2}(m-1) -k
                \big)
                \big(p-k\big)
    \\
         &= \Big( \frac{1}{2}(m-1) +k \Big)! 
            \Big(  -1  \Big)^{\frac{m-1}{2}-k}    
            \Big( \frac{1}{2}(m-1) -k \Big)! 
            \Big( p - k \Big)
    \\
\alpha  &=   \prod_{t = -\frac{1}{2}(m-1) }^{\frac{1}{2}(m-1)}  \Big(p-t\Big)
\end{split}
\end{equation*}

putting $\alpha , \beta$ back in equation \ref{alpha_beta}
\begin{equation*}
\begin{split}
l_k(x) &= 
\frac{ \alpha }{ \beta } 
\\
&=
\frac{\prod_{t = -\frac{1}{2}(m-1) }^{\frac{1}{2}(m-1)}  (p-t)}{
\Big( \frac{1}{2}(m-1) +k \Big)! 
\Big(  -1  \Big)^{\frac{m-1}{2}-k}    
\Big( \frac{1}{2}(m-1) -k \Big)! 
\Big( p - k \Big)
}
% \Big(
%  \prod_{t = -\frac{1}{2}(m-1) }^{\frac{1}{2}(m-1)}  (p-t)
% \Big)
\end{split}
\end{equation*}

\begin{equation} \label{A_final}
\begin{split}
&=
\frac{
\Big(  -1  \Big)^{m-1}    
}{
\Big(  -1  \Big)^{m-1}    
}
\frac{
\Big(  -1  \Big)^{k-\frac{m-1}{2}}    
\prod_{t = -\frac{1}{2}(m-1) }^{\frac{1}{2}(m-1)}  (p-t)
}{
\Big( \frac{1}{2}(m-1) +k \Big)! 
\Big( \frac{1}{2}(m-1) -k \Big)! 
\Big( p - k \Big)
}
\\
&=
\frac{ (-1)^{\frac{m-1}{2}+k} }{
(\frac{m-1}{2} + k)!
(\frac{m-1}{2} - k)!
(p-k)}
\prod_{t = -\frac{1}{2}(m-1) }^{\frac{1}{2}(m-1)}  (p-t)
\\
&=
\mathcal{A}_{k}^{m}(p)
\end{split}
\end{equation}

By using equations \ref{A_final} , \ref{main_eqn} , \ref{remainder_final}

$$
f(x) = f(x_0 + ph) = \sum_{k} \mathcal{A}_{k}^{m} f(x_k) + \mathcal{R}_{m-1,p}
$$
where k varies from $-\frac{1}{2}(m-1) \leq k \leq \frac{1}{2}(m-1)  \quad  \textit{,} \quad
\textit{m is odd} $.
\vspace{5mm}


Similarly, for even $m$, we can get the result by constructing Lagrange interpolating polynomial from equispaced points 
$$
x_{-\frac{1}{2}(m-2)} , x_{-\frac{1}{2}(m-2)+1} , .......
, x_{\frac{1}{2}m-1} , x_{\frac{1}{2}m}
$$

\end{proof}













%---------------------------------------------
\nocite{*}
\bibliographystyle{unsrt}
\bibliography{ref.bib}

\end{document}
