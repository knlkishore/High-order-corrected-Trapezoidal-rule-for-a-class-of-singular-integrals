\documentclass[../document.tex]{subfiles}
\begin{document}
	
	\chapter{Trapezoidal rule}
	Among all existing Quadrature rules, the \emph{Trapezoidal rule} is the most simple  and numerically stable,
	Furthermore, offer several other advantages in terms of speed of computations in many algorithms.
	
	Two major advantages of Trapezoidal rule are
	\begin{itemize}
		\item It converges to high order for smooth and periodic functions.
		\item It can be implemented with Fast Fourier Transform.
	\end{itemize}
	
	%%--------------------------------------------
	
	\section{Error in Trapezoidal rule}
	
	\begin{theorem}{Euler-Maclaurin theorem}:
		Let $a,b$ with $a<b$ be two real number and let $p\geq1$ be an integer. Then for function $f \in \mathcal{C}_p[a,b]$ ,
		there exist a real number $\xi$ in $(a,b)$ such that
		\begin{equation} \label{euler_macl}
			\int_a^b f(x)dx - \mathcal{T}_n(f) = 
			\sum_{j=1}^{ \lfloor p/2\rfloor -1 }  h^{2j} \frac{\mathcal{B}_{2j}}{2j!}
			\left[
			f^{(2j-1)}(b) - f^{(2j-1)}(a) 
			\right] -  E_n(f) 
		\end{equation}
		$$
		E_n(f) = \frac{h^p  \mathcal{B}_p  }{p!}  f^{ (p)} (\xi)
		$$
		where $\mathcal{T}_n(f)$ is $n+1$ point Trapezoidal rule, $\mathcal{B}_j$ is $jth$ Bernoulli number.
	\end{theorem}
	
	
	The error of the composite trapezoidal rule is the difference between the value of the analytical integral and the numerical result via trapezoidal rule.
	\begin{equation*}
		\begin{split}
			\mathrm{E}_n(f) =& \left| \int_a^b f(x)dx - \mathcal{T}_n (f) \right|
			\\
			=& \left| \sum_{j=1}^{ \lfloor p/2\rfloor -1 }  h^{2j} \frac{\mathcal{B}_{2j}}{2j!}
			\left[
			f^{(2j-1)}(b) - f^{(2j-1)}(a) 
			\right] -  \frac{h^p  \mathcal{B}_p  }{p!}  f^{ (p)} (\xi) \right| 
		\end{split}
	\end{equation*}
	
	\begin{lemma}
		Consider a periodic integrand $f \in \mathcal{C}_p[0,T]$ having period $T$. Then $f(0) = f(T)$. Then error in integration of $f$ over interval $[0,T]$ will be given by:
		
		$$
		\mathrm{E}_n(f)
		= \left| \sum_{j=1}^{ \lfloor p/2\rfloor -1 }  h^{2j} \frac{\mathcal{B}_{2j}}{2j!}
		\left[
		f^{(2j-1)}(T) - f^{(2j-1)}(0) 
		\right] -  \frac{h^p  \mathcal{B}_p  }{p!}  f^{ (p)} (\xi) \right|
		$$
		
		As derivative of a periodic function is always periodic with the same period.
		$f(T) = f(0) \implies f^{(j-1)}(T) = f^{(j-1)}(0)$.
		$$
		\mathrm{E}_n(f) = \left| - \frac{h^p  \mathcal{B}_p  }{p!}  f^{ (p)} (\xi)\right| 
		$$
		$$
		\mathcal{T}_n(f) = \int_0^T f(x)dx +  \mathcal{O}(h^p)
		$$
		Trapezoidal rule converges to high order for smooth and periodic functions.
	\end{lemma}
	
	%%--------------------------------------------
	
	\section{Corrected Trapezoidal rule}
	
	By Euler-Maclaurin theorem, for any function $f \in \mathcal{C}_{2p+1}[a,b]$ ,
	there exist a real number $\xi$ in $(a,b)$ such that
	\begin{equation*}
		\begin{split}
			\int_a^b f(x)dx - \mathcal{T}_n(f) &=
			\\
			&= 
			\sum_{j=1}^{p}  h^{2j} \frac{\mathcal{B}_{2j}}{(2j)!}
			\left[
			f^{(2j-1)}(b) - f^{(2j-1)}(a) 
			\right] -  E_n(f) 
			\\
			&= 
			\sum_{j=1}^{p}  h^{2j}
			\left[\frac{\mathcal{B}_{2j}}{(2j)!}
			\left(
			f^{(2j-1)}(b) - f^{(2j-1)}(a) 
			\right) \right] 
			-\frac{h^{p+1}  \mathcal{B}_{p+1}  }{p!}  
			f^{ (p+1)} (\xi)
			\\
			&=\alpha_1h^2 +\alpha_2h^4 + .... + 
			\alpha_k h^{2p} + \mathcal{O}_n(h^{2p+1})
		\end{split}
	\end{equation*}
	
	here $\alpha_j =\frac{ \mathcal{B}_{j} }{(2j)!}\left[ f^{2j-1}(b) - f^{2j-1}(a) \right] $.
	
	$$
	\int_a^b f(x)dx - \mathcal{T}_n(f) =\alpha_1h^2 +\alpha_2h^4 + .... + \alpha_k h^{2p} + \mathcal{O}_n(h^{2p+1})
	$$
	
	In case for some functions if a function's derivative is not explicitly known, then we must approximate $f'(x)$ , one way to do it via Taylor's theorem.
	\\For example
	$$
	\mathcal{D}f(a) = 
	\frac{-3f(a) + 4f(a+h) -f(a+2h)}{2h} = f'(a) +\mathcal{O}(h^2)
	$$
	
	A standard endpoint corrected trapezoidal is defined by $\mathcal{CT}_n$ is given by.
	
	$$
	\mathcal{CT}_n(f)  = \mathcal{T}_n(f) + \frac{1}{12}
	\left[
	f'(b)  - f'(a)
	\right] h^2
	$$
	A standard endpoint corrected trapezoidal rule is fourth-order convergent $\mathcal{O}_n(h^4)$
	in case  $f'(x)$ is not explicitly known , we can use approximated
	values of $f'(x)$ .
	
	\vspace{3mm} %5mm vertical space
	
	
	Similarly, we can also define High-Order Endpoint Corrected Trapezoidal rule as
	For $m\geq 1$ ,
	$\alpha = (\alpha_1,\alpha_2,....,\alpha_m)$.
	
	
	$\mathcal{T}_n^{\alpha}$. 
	$$\mathcal{T}_n^{\alpha} (f) =  \mathcal{T}_n(f) +\alpha_1h^2 +.....
	+\alpha_m h^{2m}
	$$ 
	it will have  error bound of $ \mathcal{O}(h^{2m+1})$.
	
	$\alpha = (\alpha_1,\alpha_2,....,\alpha_m)$, can be calculated without involving explicit derivatives of functions, using methods like Taylor's theorem and others. 
	
	\section{Pre-corrected Trapezoidal rule}
	Consider an integral of type.
	$$
	\int_a^b K(x-y)\varphi(x)dx
	$$
	where Kernel $K(x-y)$ has singularity at $x=y$. Pre-corrected Trapezoidal rule for this integrand is
	\begin{multline*}
		\int_a^b K(x-y)\varphi(x)dx =\\ 
		\frac{h}{2}
		\left[
		K(a-y)\varphi(a) +  K(b-y)\varphi(b) +
		2\sum_{\substack{  	j=1 \\x_j \neq y}}^{n-1}
		K(x_j-y)\varphi(x_j)
		\right]  + h^2 w_{pc} 
	\end{multline*}
	
	here $w_{pc}$ is a pre-calculated weights depending on kernel $K(x-y)$. In 
	Pre-corrected Trapezoidal rule, we are taking the standard composite Trapezoidal rule and removing the term containing the integrand value at the point of singularity and add a term containing a pre-computed value,$ w_{pc}$.
	
	%----------------------------------------------------------------------------
	%----------------------------------------------------------------------------
	\chapter{Plan for the next semester}
	Our objective will be to use a corrected Trapezoidal rule to compute convolution integrals of the form
	$$
	A[\phi](x) = \int_{\Omega}K(x-y)\phi(y)dy, \ \ \text{for} \  x\in \Omega.	
	$$
	
	Such integrals are used to evaluate:
	
	\begin{itemize}
		\item {Laplace Kernel} (Appears in numerous applications, e.g., Fluid dynamics, Electrostatics, Gravitational potentials....   )
		$$
		K(x-y):=\left\{\begin{array}{ll}
			\frac{1}{2 \pi} \ln \frac{1}{|x-y|}, & \text{in the two dimensions} \\
			\frac{1}{4 \pi} \frac{1}{|x-y|}, & \text{in the three dimensions} 
		\end{array}\right.
		$$
		
		\item {Helmholtz kernel} (Appears in the context of wave scattering problems)
		$$
		K(x-y):=\left\{\begin{array}{ll}
			\frac{i}{4 \pi} H^{1}_{0}|x-y|, & \text{in the two dimensions} \\
			\frac{1}{4 \pi} \frac{e^{i\kappa|x-y|}}{4\pi |x-y|}, & \text{in the three dimensions} 
		\end{array}\right.
		$$
	\end{itemize}
	
	
	\vspace{5mm} %5mm vertical space
	Also pre-corrected trapezoidal rule for the logarithmic singularity in the two dimension reported in:\newline
	\texttt{\textbf{
			J.C. Aguilar, Y. Chen, High-order corrected trapezoidal quadrature rules for the Coulomb potential in three dimensions, Comput. Math. Appl. 49 (4) (2005) 625–631.
	}}\newline
	that will also be an area of study.
	\vspace{5mm} %5mm vertical space
	
	
	Also, we will study, a new technique for the computation of pre-corrected weights is reported in: 
	
	\texttt{\textbf{B.Wu and P.-G.Martinsson,
			Zeta correction: A new approach to constructing corrected trapezoidal quadrature rules for singular integral operators.
			arXiv preprint arXiv:2007.13898, 2020.
	}}
	
	
	
\end{document}